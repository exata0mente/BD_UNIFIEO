% ----------------------------------------------------------
% ELEMENTOS TEXTUAIS
% ----------------------------------------------------------

% Fazer as citações!!!!!!!!!!


\textual

% ----------------------------------------------------------
% Introdução (exemplo de capítulo sem numeração, mas presente no Sumário)
% ----------------------------------------------------------
\chapter*[Introdução]{Introdução}
\addcontentsline{toc}{chapter}{Introdução}

A facilidade de acesso à diversas formas de pagamento, incentivadas pelo governo no intuito de acelerar a economia, deu grande poder de compra ao consumidor brasileiro. Apesar das boas oportunidades que surgem, ainda não estamos habituados a controlar nossos gastos. 

Conforme Barbosa, Silva e Prado (2012, apud FECOMERCIO, 2009) a falta de planejamento, face aos gastos rotineiros combinados àqueles movidos por impulso/oportunidade, pode desencadear um desequilíbrio que acaba por impactar na saúde financeira de diversas cadeias de negócios.

Felizmente, ferramentas que auxiliam no planejamento financeiro estão bem difundidas na grande rede, além de serem fáceis de montar com um editor de planilha.
Ainda em Barbosa, Silva e Prado (2012) “O processo de planejamento é a ferramenta que as pessoas e organizações usam para administrar suas relações com o futuro. É uma aplicação específica do processo decisório visto que as decisões que podem, de alguma forma influenciar o futuro, ou que serão colocadas em prática no futuro, são decisões de planejamento (apud MAXIMIANO, 2005).

Oliveira (2010) ensina que “para se elaborar um orçamento é preciso anotar todas as receitas e todas as despesas feitas por alguns meses. As receitas são o salário líquido e todas as outras possíveis fontes de renda possíveis como aluguéis, aposentadoria, pensão, juros de rendimentos etc. Normalmente as pessoas se preocupam mais com as grandes despesas, aquelas fáceis de perceber, e que podem ser consideradas como as despesas das quais se tem certeza de que irá ocorrer como a prestação da casa e do carro, a escola dos filhos, as compras de supermercados e outras, no entanto as pessoas se descuidam das pequenas despesas, como gastos com lazer, lanches, presentes e outras, fato que afeta o comprometimento da renda e o nível de endividamento.”

Desta forma, o intuito deste projeto é desenvolver um banco de dados, estruturado com os conceitos básicos de um orçamento familiar (relação receitas x custos) que possa ser a base de uma aplicação que consiga auxiliar na tomada de decisão de uma família, mostrando média de consumo, projeções para o decorrer do ano (semana, mês, trimestre), perfil de gastos, comparativos de períodos e outras formas de apresentação.
